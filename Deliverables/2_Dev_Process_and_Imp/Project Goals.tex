\documentclass[10pt,letterpaper]{article}
\setlength{\paperheight}{11in}
\setlength{\paperwidth}{8.5in}
\usepackage[ascii]{inputenc}
\usepackage{amsmath}
\usepackage{amsfonts}
\usepackage{amssymb}
\usepackage{graphicx}
\usepackage{enumitem}
\usepackage{soul}
\usepackage{array}
\newcolumntype{C}[1]{>{\centering\arraybackslash}p{#1}}
\usepackage{latexsym}
\usepackage{hyperref} % Inserts hyper-references in the table of contents
\hypersetup{
	colorlinks,
	citecolor=black,
	filecolor=black,
	linkcolor=black,
	urlcolor=black
}
\usepackage{tikz} 			% Diagrams
\usetikzlibrary{arrows}		% Diagrams
\usepackage{verbatim}		% Diagrams
\tikzstyle{int}=[draw, fill=blue!20, minimum size=2em]
\tikzstyle{init} = [pin edge={to-,thin,black}]


\begin{document}
	
	% TITLE PAGE
	
	\begin{titlepage}
		\newcommand{\HRule}{\rule{\linewidth}{0.5mm}}
		\center
		
		\textsc{\LARGE McMaster University}\\[1.5cm] % Name of your university/college
		\textsc{\Large Project Goals}\\[0.5cm] % Major heading such as course name
		\textsc{\large 4G06 Capstone Design Project}\\[0.5cm] % Minor heading such as course title
		
		\HRule \\[0.4cm]
		{ \huge \bfseries 3D Scanning Quadcopter \\[2mm] \textit{A Large Scale 3D Scanner}}\\[0.4cm] % Title of your document
		\HRule \\[1.5cm]
		
		\begin{tabular}{ccc}
			\bf{Paul Correia}		& \bf{Nicolas Lelievre} 	& \bf{Bennett Mackenzie}		\\
			Mechatronics Eng 		& Software Eng 				& Software Eng 					\\
			\textit{1132370} 		& \textit{1203446}			& \textit{1211985} 				\\ \\
			\bf{Tigran Martikan} 	& \bf{Balraj Shah} 			& \bf{Mykola Somov} 			\\
			Software Eng			& Software Eng				& Software Eng 					\\
			\textit{1213170} 		& \textit{1207997}			& \textit{1141160}
		\end{tabular}\\[4cm]
		
		{\large October 19, 2015}\\[3cm] 
		
		%\includegraphics{Logo}\\[1cm] % Include a department/university logo - this will require the graphicx package
		
		\vfill % Fill the rest of the page with whitespace
		
	\end{titlepage}
	
	
\tableofcontents

\newpage
	

\section{Project Goals}
Below is a list of project goals for the 3D Scanning Quadcopter, a large scale 3D scanner. 

\begin{enumerate}
	\item The 3D Scanning Quadcopter will be able to scan an object of size up to 6 meters in height, 6 meters in length, and 6 meters in width. 
	\item The 3D Scanning Quadcopter will be fully autonomous, and will be able to scan an object without any input from a human aside from hitting ``Start''. 
	\item The 3D Scanning Quadcopter will be able to complete the scanning of a 3D object in a maximum of 15 minutes (for objects of the maximum size). 
	\item The 3D Scanning Quadcopter will be able to produce a model file of the scanned object that is 3D printable.
	\item The 3D Scanning Quadcopter system should be portable and deployable by a single person.
\end{enumerate}

\subsection{Reasoning}
3D scanning technology is something that already exists, as many already know. However, we've identified that 3D scanning technology that is both flexible and autonomous isn't currently on the market. Autonomous scanners currently in existence are very rigid in terms of what objects can be scanned. The objects must fit on a fixed platform for the machine to scan them, which means that the larger an object, the larger the scanning apparatus must be, and after a certain size threshold having autonomous 3D scanner that could scan an object would be inconvenient at best and infeasible at worst. \par 

More flexible 3D scanning technology exists, but it is limited by the human factor. Scanners exist that let humans scan an object manually from different angles, allowing a 3D model to be produced based on those scans. This solves the flexibility problem: having a hand-held scanner allows a human to scan 3D objects of various sizes without having a bulky apparatus. However, this introduces possible human error in the scanning, and limits the size of the object being scanned to that which a human can reach. \par 

A 3D Scanning Quadcopter solves all these problems. It is autonomous and therefore removes human error in the scanning of objects, it can scan objects of both larger and smaller sizes while still remaining portable, as well as scan things that a human is not able to (perhaps due to the object being too tall or irregular in shape).





\end{document}